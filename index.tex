% Options for packages loaded elsewhere
\PassOptionsToPackage{unicode}{hyperref}
\PassOptionsToPackage{hyphens}{url}
%
\documentclass[
  letterpaper,
]{book}

\usepackage{amsmath,amssymb}
\usepackage{iftex}
\ifPDFTeX
  \usepackage[T1]{fontenc}
  \usepackage[utf8]{inputenc}
  \usepackage{textcomp} % provide euro and other symbols
\else % if luatex or xetex
  \usepackage{unicode-math}
  \defaultfontfeatures{Scale=MatchLowercase}
  \defaultfontfeatures[\rmfamily]{Ligatures=TeX,Scale=1}
\fi
\usepackage{lmodern}
\ifPDFTeX\else  
    % xetex/luatex font selection
\fi
% Use upquote if available, for straight quotes in verbatim environments
\IfFileExists{upquote.sty}{\usepackage{upquote}}{}
\IfFileExists{microtype.sty}{% use microtype if available
  \usepackage[]{microtype}
  \UseMicrotypeSet[protrusion]{basicmath} % disable protrusion for tt fonts
}{}
\makeatletter
\@ifundefined{KOMAClassName}{% if non-KOMA class
  \IfFileExists{parskip.sty}{%
    \usepackage{parskip}
  }{% else
    \setlength{\parindent}{0pt}
    \setlength{\parskip}{6pt plus 2pt minus 1pt}}
}{% if KOMA class
  \KOMAoptions{parskip=half}}
\makeatother
\usepackage{xcolor}
\setlength{\emergencystretch}{3em} % prevent overfull lines
\setcounter{secnumdepth}{5}
% Make \paragraph and \subparagraph free-standing
\makeatletter
\ifx\paragraph\undefined\else
  \let\oldparagraph\paragraph
  \renewcommand{\paragraph}{
    \@ifstar
      \xxxParagraphStar
      \xxxParagraphNoStar
  }
  \newcommand{\xxxParagraphStar}[1]{\oldparagraph*{#1}\mbox{}}
  \newcommand{\xxxParagraphNoStar}[1]{\oldparagraph{#1}\mbox{}}
\fi
\ifx\subparagraph\undefined\else
  \let\oldsubparagraph\subparagraph
  \renewcommand{\subparagraph}{
    \@ifstar
      \xxxSubParagraphStar
      \xxxSubParagraphNoStar
  }
  \newcommand{\xxxSubParagraphStar}[1]{\oldsubparagraph*{#1}\mbox{}}
  \newcommand{\xxxSubParagraphNoStar}[1]{\oldsubparagraph{#1}\mbox{}}
\fi
\makeatother


\providecommand{\tightlist}{%
  \setlength{\itemsep}{0pt}\setlength{\parskip}{0pt}}\usepackage{longtable,booktabs,array}
\usepackage{calc} % for calculating minipage widths
% Correct order of tables after \paragraph or \subparagraph
\usepackage{etoolbox}
\makeatletter
\patchcmd\longtable{\par}{\if@noskipsec\mbox{}\fi\par}{}{}
\makeatother
% Allow footnotes in longtable head/foot
\IfFileExists{footnotehyper.sty}{\usepackage{footnotehyper}}{\usepackage{footnote}}
\makesavenoteenv{longtable}
\usepackage{graphicx}
\makeatletter
\newsavebox\pandoc@box
\newcommand*\pandocbounded[1]{% scales image to fit in text height/width
  \sbox\pandoc@box{#1}%
  \Gscale@div\@tempa{\textheight}{\dimexpr\ht\pandoc@box+\dp\pandoc@box\relax}%
  \Gscale@div\@tempb{\linewidth}{\wd\pandoc@box}%
  \ifdim\@tempb\p@<\@tempa\p@\let\@tempa\@tempb\fi% select the smaller of both
  \ifdim\@tempa\p@<\p@\scalebox{\@tempa}{\usebox\pandoc@box}%
  \else\usebox{\pandoc@box}%
  \fi%
}
% Set default figure placement to htbp
\def\fps@figure{htbp}
\makeatother

\usepackage[a4paper,margin=2cm]{geometry}
\usepackage{fancyhdr}
\pagestyle{fancy}
\fancyhead{}  % Borra todos los headers
\renewcommand{\headrulewidth}{0pt}
\fancyfoot[C]{\thepage}
\makeatletter
\@ifpackageloaded{bookmark}{}{\usepackage{bookmark}}
\makeatother
\makeatletter
\@ifpackageloaded{caption}{}{\usepackage{caption}}
\AtBeginDocument{%
\ifdefined\contentsname
  \renewcommand*\contentsname{Tabla de contenidos}
\else
  \newcommand\contentsname{Tabla de contenidos}
\fi
\ifdefined\listfigurename
  \renewcommand*\listfigurename{Listado de Figuras}
\else
  \newcommand\listfigurename{Listado de Figuras}
\fi
\ifdefined\listtablename
  \renewcommand*\listtablename{Listado de Tablas}
\else
  \newcommand\listtablename{Listado de Tablas}
\fi
\ifdefined\figurename
  \renewcommand*\figurename{Figura}
\else
  \newcommand\figurename{Figura}
\fi
\ifdefined\tablename
  \renewcommand*\tablename{Tabla}
\else
  \newcommand\tablename{Tabla}
\fi
}
\@ifpackageloaded{float}{}{\usepackage{float}}
\floatstyle{ruled}
\@ifundefined{c@chapter}{\newfloat{codelisting}{h}{lop}}{\newfloat{codelisting}{h}{lop}[chapter]}
\floatname{codelisting}{Listado}
\newcommand*\listoflistings{\listof{codelisting}{Listado de Listados}}
\makeatother
\makeatletter
\makeatother
\makeatletter
\@ifpackageloaded{caption}{}{\usepackage{caption}}
\@ifpackageloaded{subcaption}{}{\usepackage{subcaption}}
\makeatother

\ifLuaTeX
\usepackage[bidi=basic]{babel}
\else
\usepackage[bidi=default]{babel}
\fi
\babelprovide[main,import]{spanish}
% get rid of language-specific shorthands (see #6817):
\let\LanguageShortHands\languageshorthands
\def\languageshorthands#1{}
\usepackage{bookmark}

\IfFileExists{xurl.sty}{\usepackage{xurl}}{} % add URL line breaks if available
\urlstyle{same} % disable monospaced font for URLs
\hypersetup{
  pdftitle={Nuevas Tendencias en la Rehabilitación},
  pdfauthor={Equipo de Cátedra --- Licenciatura en Kinesiología y Fisiatría},
  pdflang={es},
  hidelinks,
  pdfcreator={LaTeX via pandoc}}


\title{Nuevas Tendencias en la Rehabilitación}
\author{Equipo de Cátedra --- Licenciatura en Kinesiología y Fisiatría}
\date{2024-12-31}

\begin{document}
\frontmatter
\maketitle

\renewcommand*\contentsname{Tabla de contenidos}
{
\setcounter{tocdepth}{2}
\tableofcontents
}

\mainmatter
\bookmarksetup{startatroot}

\chapter{Introducción}\label{introducciuxf3n}

\textbf{Propuesta pedagógica y manual práctico para la formación
investigativa en Kinesiología y Fisiatría}

Este eBook está dirigido a docentes universitarios y estudiantes de
cuarto año de la Licenciatura en Kinesiología y Fisiatría, como recurso
central del taller optativo ``Nuevas Tendencias en la Rehabilitación''.

\begin{center}\rule{0.5\linewidth}{0.5pt}\end{center}

La formación en investigación es un desafío recurrente en las carreras
de salud. Los futuros profesionales necesitan conectar los saberes
metodológicos aprendidos con la realidad de la producción científica.
Sin embargo, esa conexión suele quedar relegada a lo teórico, sin
espacios prácticos donde puedan experimentar cómo se investiga, se
analiza y se produce conocimiento relevante.

Este documento surge como una respuesta concreta a esa necesidad. Es a
la vez una \textbf{propuesta pedagógica para docentes} y un
\textbf{manual práctico para estudiantes}. Su objetivo es guiar el
desarrollo del taller optativo \emph{Nuevas Tendencias en la
Rehabilitación}, brindando una experiencia auténtica de revisión
sistemática de alcance sobre la producción científica reciente en
fisioterapia y kinesiología en América Latina y Brasil.

La obra está estructurada en capítulos pensados para acompañar a ambos
públicos, articulando fundamentos, herramientas y ejemplos claros. Se
promueve el trabajo colaborativo, la autonomía crítica y la
participación activa, con el propósito de fortalecer la alfabetización
científica y la producción académica significativa.

\begin{center}\rule{0.5\linewidth}{0.5pt}\end{center}

\section{¿A quiénes está dirigido este
eBook?}\label{a-quiuxe9nes-estuxe1-dirigido-este-ebook}

\begin{itemize}
\tightlist
\item
  \textbf{Docentes:} encontrarán fundamentos, objetivos y una
  organización didáctica argumentada, junto a criterios de evaluación y
  sugerencias metodológicas.
\item
  \textbf{Estudiantes:} dispondrán de una guía para participar en una
  revisión sistemática, con ejemplos, consejos prácticos y un espacio
  colaborativo de intercambio y aprendizaje.
\end{itemize}

\begin{center}\rule{0.5\linewidth}{0.5pt}\end{center}

\section{Estructura del eBook}\label{estructura-del-ebook}

\begin{enumerate}
\def\labelenumi{\arabic{enumi}.}
\tightlist
\item
  \textbf{Fundamento y propósito del taller}\\
  Justificación, objetivos y organización didáctica.
\item
  \textbf{El proyecto de investigación y modelo de extracción de
  datos}\\
  Herramientas y variables clave para el análisis de la literatura
  científica.
\item
  \textbf{Guía práctica para estudiantes}\\
  Orientación paso a paso, ejemplos y recomendaciones.
\item
  \textbf{El proceso colaborativo - Wiki como herramienta}\\
  Registro de dudas, errores, aprendizajes y recomendaciones para
  futuras cohortes.
\end{enumerate}

\begin{center}\rule{0.5\linewidth}{0.5pt}\end{center}

\section{Nota sobre alcances y
limitaciones}\label{nota-sobre-alcances-y-limitaciones}

Reconocemos que la centralidad de la investigación y la distancia entre
teoría y práctica no se presentan del mismo modo en todas las carreras
ni instituciones. La elección de la revisión sistemática de alcance como
eje del taller responde, ante todo, a la práctica habitual de la mayoría
de los estudiantes ---quienes suelen optar por revisiones
bibliográficas--- y a criterios de viabilidad dentro del marco
institucional actual. Asumimos que este material es perfectible y que el
intento de dialogar a la vez con docentes y estudiantes presenta
desafíos que solo podrán resolverse con la experiencia y el aporte de
cada cohorte. Invitamos a quienes utilicen este eBook a considerarlo
como un punto de partida.

\bookmarksetup{startatroot}

\chapter{Fundamento y propósito}\label{fundamento-y-propuxf3sito}

\section{Contexto}\label{contexto}

La enseñanza de la investigación en las carreras de salud suele
enfrentar el desafío de articular los contenidos metodológicos con las
prácticas reales de producción de conocimiento. En la Licenciatura en
Kinesiología y Fisiatría, esta dificultad se expresa en la distancia
frecuente entre los contenidos teóricos de metodología de la
investigación y su aplicación concreta en contextos situados.

Aunque los planes de estudio incluyen espacios destinados a la formación
metodológica, es habitual observar que una parte de los estudiantes
tiene dificultades para transferir esos aprendizajes al momento de
realizar trabajos de investigación propios, especialmente en la
instancia del trabajo final de grado. Por supuesto, la magnitud y
características de esta brecha pueden variar según la institución, la
trayectoria individual y la cultura pedagógica, por lo que no debe
asumirse como un problema universal ni estático.

Este taller optativo, ``Nuevas Tendencias en la Rehabilitación'', se
concibe como un espacio de articulación entre saberes previos y la
experiencia concreta de participación en un proceso investigativo
grupal, orientado a la revisión sistemática de alcance de la producción
científica regional más reciente. Esta elección metodológica responde
tanto a la práctica habitual de los estudiantes ---que suelen optar por
revisiones bibliográficas--- como a criterios de viabilidad dentro de
las condiciones institucionales actuales. Es importante señalar que la
revisión sistemática no es el único camino posible para articular teoría
y práctica, ni pretende sustituir otras experiencias valiosas de
investigación (como estudios de caso, proyectos comunitarios,
investigación cualitativa, etc.), pero constituye en este contexto una
alternativa pertinente y factible.

La propuesta busca mejorar competencias metodológicas y analíticas,
estimular el trabajo colaborativo y fomentar la reflexión crítica sobre
las tendencias en la disciplina, así como la producción de aportes
relevantes para el campo profesional y académico. Sin embargo,
reconocemos que el logro de estos objetivos depende de múltiples
factores: el interés y la motivación de los estudiantes, el
acompañamiento docente y las condiciones institucionales. Por eso, este
taller se concibe como una experiencia en construcción, abierta a la
mejora continua y a la incorporación del \emph{feedback} de cada
cohorte.

\begin{quote}
\textbf{Nota para los participantes}

Invitamos tanto a estudiantes como a docentes a considerar este taller
como un punto de partida perfectible, donde la experiencia y la
reflexión compartida pueden enriquecer la propuesta año a año. Sus
aportes y sugerencias serán fundamentales para fortalecer el puente
entre la formación metodológica y la práctica real de la investigación
en salud.
\end{quote}

\begin{center}\rule{0.5\linewidth}{0.5pt}\end{center}

\section{Objetivos y justificación}\label{objetivos-y-justificaciuxf3n}

\textbf{Objetivo general:}\\
Promover la alfabetización científica, el desarrollo de competencias
metodológicas y el análisis critico de la bibliografía en los
estudiantes, a través de una experiencia guiada y colaborativa de
revisión sistemática de alcance, orientada a mapear la producción
científica reciente en fisioterapia y kinesiología en América Latina y
Brasil.

\textbf{Objetivos específicos:}\\
- Articular saberes metodológicos previos con una práctica investigativa
concreta. - Potenciar el trabajo colaborativo, la autonomía crítica y la
producción académica significativa. - Estimular la reflexión sobre
tendencias, vacíos y desafíos actuales en la disciplina. - Generar un
aporte colectivo y transferible a futuras cohortes mediante la
construcción de un registro colaborativo.

\textbf{Justificación:}\\
El taller responde a una necesidad concreta observada en la formación:
la dificultad para trasladar los conocimientos teóricos en investigación
a prácticas reales y contextualizadas. Al proponer una revisión
sistemática de alcance centrada en la producción regional, se busca no
solo fortalecer competencias técnicas, sino también fomentar el
pensamiento crítico, la colaboración y la participación activa en la
comunidad académica y profesional.

\begin{center}\rule{0.5\linewidth}{0.5pt}\end{center}

\section{Organización didáctica}\label{organizaciuxf3n-diduxe1ctica}

El taller se organiza en torno a las siguientes instancias principales:

\begin{itemize}
\tightlist
\item
  \textbf{Introducción y contextualización:} presentación del taller,
  repaso de saberes previos y encuadre de la revisión sistemática de
  alcance.
\item
  \textbf{Formación de equipos y definición de roles:} distribución de
  tareas, acuerdo sobre criterios y organización del trabajo
  colaborativo.
\item
  \textbf{Búsqueda y selección de estudios:} aplicación de estrategias
  de búsqueda, lectura crítica y aplicación de criterios de
  inclusión/exclusión.
\item
  \textbf{Extracción y registro de datos:} uso del modelo de extracción
  consensuado, organización de la base de datos y primeras
  aproximaciones al análisis.
\item
  \textbf{Análisis y síntesis:} discusión grupal de hallazgos,
  elaboración de gráficos y tablas descriptivas, identificación de
  vacíos temáticos.
\item
  \textbf{Producción del informe:} redacción del documento final de
  revisión y aporte al registro colaborativo wiki, destinado a futuras
  cohortes.
\end{itemize}

El cronograma y la distribución temporal de actividades se adaptarán a
la dinámica y necesidades del grupo, priorizando el avance sostenido y
el trabajo reflexivo.

\begin{center}\rule{0.5\linewidth}{0.5pt}\end{center}

\section{Cronograma de actividades tentativo (14
clases)}\label{cronograma-de-actividades-tentativo-14-clases}

\begin{longtable}[]{@{}
  >{\raggedright\arraybackslash}p{(\linewidth - 4\tabcolsep) * \real{0.0574}}
  >{\raggedright\arraybackslash}p{(\linewidth - 4\tabcolsep) * \real{0.5656}}
  >{\raggedright\arraybackslash}p{(\linewidth - 4\tabcolsep) * \real{0.3770}}@{}}
\toprule\noalign{}
\begin{minipage}[b]{\linewidth}\raggedright
Clase
\end{minipage} & \begin{minipage}[b]{\linewidth}\raggedright
Tema / Actividad principal
\end{minipage} & \begin{minipage}[b]{\linewidth}\raggedright
Objetivos específicos
\end{minipage} \\
\midrule\noalign{}
\endhead
\bottomrule\noalign{}
\endlastfoot
1 & Presentación del taller, encuadre y objetivos & Conocer la
propuesta, formar equipos \\
2 & Repaso de saberes previos, introducción a revisiones de alcance &
Unificar criterios, aclarar dudas \\
3 & Estrategias de búsqueda bibliográfica & Elaborar estrategias y
términos de búsqueda \\
4 & Aplicación de criterios de inclusión y exclusión & Primer filtrado
de artículos \\
5 & Lectura crítica de artículos seleccionados & Discusión y acuerdos
sobre calidad \\
6 & Introducción al modelo de extracción de datos & Definir variables y
consensuar el uso \\
7 & Extracción de datos (1° parte) & Iniciar carga de datos
colaborativa \\
8 & Extracción de datos (2° parte) & Continuar y depurar base de
datos \\
9 & Organización y análisis preliminar de resultados & Comenzar a
identificar tendencias \\
10 & Elaboración de tablas y gráficos descriptivos & Visualizar
información, detectar vacíos \\
11 & Discusión grupal de hallazgos y vacíos temáticos & Compartir
aprendizajes, nuevas preguntas \\
12 & Redacción de informe preliminar y aportes al wiki & Sistematizar el
trabajo, compartir consejos \\
13 & Revisión final del informe y del registro wiki & Ajustar y mejorar
los productos finales \\
14 & Socialización de resultados y cierre del taller & Reflexión
colectiva, evaluación y despedida \\
\end{longtable}

\begin{center}\rule{0.5\linewidth}{0.5pt}\end{center}

\section{Evaluación}\label{evaluaciuxf3n}

La evaluación contempla tanto la participación activa y sostenida de
cada estudiante como la calidad del trabajo colaborativo y de los
productos generados. Los principales criterios de evaluación son:

\begin{itemize}
\tightlist
\item
  Asistencia y participación en las distintas etapas del taller.
\item
  Aportes individuales y grupales a la revisión y al documento
  colaborativo.
\item
  Calidad metodológica y presentación del informe final.
\item
  Contribuciones al registro wiki como legado para futuras cohortes.
\end{itemize}

Se valorará especialmente la disposición al trabajo en equipo, la
capacidad de argumentar decisiones metodológicas y la actitud reflexiva
y propositiva ante los desafíos del proceso.

\begin{center}\rule{0.5\linewidth}{0.5pt}\end{center}

\bookmarksetup{startatroot}

\chapter{El proyecto de investigación y el modelo de extracción de
datos}\label{el-proyecto-de-investigaciuxf3n-y-el-modelo-de-extracciuxf3n-de-datos}

Este capítulo presenta el marco general del proyecto de investigación
que estructura el taller, junto con las herramientas metodológicas y el
modelo de extracción que guiarán todo el proceso.

\begin{center}\rule{0.5\linewidth}{0.5pt}\end{center}

\section{Planteo general del
proyecto}\label{planteo-general-del-proyecto}

\subsection{Pregunta central}\label{pregunta-central}

\begin{quote}
¿Qué se ha investigado en fisioterapia y kinesiología en América Latina
y Brasil entre 2023 y 2027?
\end{quote}

\subsection{Objetivos}\label{objetivos}

\textbf{General:}\\
Mapear sistemáticamente la evidencia reciente sobre fisioterapia y
kinesiología en América Latina y Brasil, identificando: - áreas clínicas
abordadas, - tipos de estudios publicados, - poblaciones analizadas, - y
vacíos temáticos o tópicos poco representados.

\textbf{Específicos:}\\
- Identificar las áreas clínicas predominantes en la producción
regional. - Analizar los tipos de diseño y metodologías más frecuentes.
- Describir las poblaciones estudiadas. - Detectar temas emergentes o
escasamente abordados.

\subsection{Diseño}\label{diseuxf1o}

\begin{itemize}
\tightlist
\item
  Revisión sistemática de alcance (\emph{scoping review}).
\item
  Basada en las directrices del Joanna Briggs Institute (JBI).
\item
  Reportada según la guía PRISMA-ScR (2020).
\end{itemize}

\begin{center}\rule{0.5\linewidth}{0.5pt}\end{center}

\section{Criterios del proyecto}\label{criterios-del-proyecto}

\subsection{Criterios de inclusión}\label{criterios-de-inclusiuxf3n}

\begin{itemize}
\tightlist
\item
  Años: 2023--2027
\item
  Idiomas: español, portugués, inglés
\item
  Autores con afiliación institucional en América Latina o Brasil
\item
  Estudios con rigurosidad metodológica básica (descripta en el trabajo)
\item
  Artículos originales, estudios cualitativos, revisiones sistemáticas,
  estudios de caso
\end{itemize}

\subsection{Criterios de exclusión}\label{criterios-de-exclusiuxf3n}

\begin{itemize}
\tightlist
\item
  Editoriales, comentarios, cartas al editor
\item
  Trabajos con deficiencias metodológicas graves o ausencia de
  justificación del diseño
\item
  Trabajos a los que no pueda accederse a su texto en forma completa
\item
  Trabajos duplicados
\end{itemize}

\begin{center}\rule{0.5\linewidth}{0.5pt}\end{center}

\section{Fuentes y estrategias de
búsqueda}\label{fuentes-y-estrategias-de-buxfasqueda}

Se utilizarán fuentes y bases de datos reconocidas en el área de la
salud, priorizando la literatura científica regional:

\begin{itemize}
\tightlist
\item
  PubMed
\item
  SciELO
\item
  LILACS (vía BVS)
\item
  PEDro
\item
  Scopus
\item
  Google Scholar / Semantic Scholar (literatura gris)
\end{itemize}

\textbf{Ejemplo de estrategia de búsqueda en PubMed:}\\
((``Physical Therapy''{[}Title/Abstract{]} OR ``Physiotherapy'' OR
``Fisioterapia'' OR ``Kinesiology'') AND (``Latin
America''{[}Affiliation{]} OR ``Brazil''{[}Affiliation{]} OR
``Argentina'' OR ``Chile'' OR ``Mexico'' OR ``Colombia'' OR ``Peru'')
AND (``2023/01/01''{[}Date - Publication{]} : ``2026/12/31''{[}Date -
Publication{]}))

\begin{center}\rule{0.5\linewidth}{0.5pt}\end{center}

\section{Selección y extracción de
datos}\label{selecciuxf3n-y-extracciuxf3n-de-datos}

\begin{itemize}
\tightlist
\item
  \textbf{Selección de estudios:}\\
  Dos revisores independientes (estudiantes) seleccionarán los artículos
  relevantes, con apoyo docente en caso de discrepancias.
\item
  \textbf{Extracción de datos:}\\
  Se realizará en base al modelo consensuado (ver siguiente apartado).
  Si surgen dudas sobre la categorización de alguna variable, se
  resolverán en grupo.
\end{itemize}

\begin{center}\rule{0.5\linewidth}{0.5pt}\end{center}

\section{Modelo de extracción de
datos}\label{modelo-de-extracciuxf3n-de-datos}

La tabla siguiente define las variables mínimas a registrar en cada
artículo:

\begin{longtable}[]{@{}
  >{\raggedright\arraybackslash}p{(\linewidth - 4\tabcolsep) * \real{0.2411}}
  >{\raggedright\arraybackslash}p{(\linewidth - 4\tabcolsep) * \real{0.4464}}
  >{\raggedright\arraybackslash}p{(\linewidth - 4\tabcolsep) * \real{0.3125}}@{}}
\toprule\noalign{}
\begin{minipage}[b]{\linewidth}\raggedright
Variable
\end{minipage} & \begin{minipage}[b]{\linewidth}\raggedright
Descripción
\end{minipage} & \begin{minipage}[b]{\linewidth}\raggedright
Formato sugerido
\end{minipage} \\
\midrule\noalign{}
\endhead
\bottomrule\noalign{}
\endlastfoot
ID & Código interno asignado al artículo & Texto (ej. A001) \\
Título & Título completo del artículo & Texto \\
Autores & Apellidos e iniciales de los autores principales & Texto \\
Año & Año de publicación & Numérico (ej. 2023) \\
País & País de afiliación del primer autor & Texto \\
Revista & Nombre de la revista & Texto \\
Tipo de estudio & Diseño metodológico principal & Texto (ej.
experimental, revisión) \\
Área temática & Campo o subdisciplina principal & Texto / Lista
predefinida \\
Población & Grupo estudiado (ej. adultos mayores, niños) & Texto \\
Objetivo principal & Objetivo central del estudio & Texto breve \\
Etiquetas (keywords) & Palabras clave, etiquetas asignadas & Texto /
Lista separada por comas \\
Fuente & Base de datos o repositorio donde se halló & Texto \\
\end{longtable}

Se registrar los datos en una planilla colaborativa (Excel, Google
Sheets, etc.), que permita el control cruzado y la revisión grupal.

\begin{center}\rule{0.5\linewidth}{0.5pt}\end{center}

\section{Reflexión pedagógica sobre las variables y el
proceso}\label{reflexiuxf3n-pedaguxf3gica-sobre-las-variables-y-el-proceso}

La selección de variables responde a la lógica de la pregunta central y
los objetivos del proyecto: permite mapear el campo, identificar
tendencias y comparar estudios entre sí. Es fundamental entender que
toda decisión sobre qué registrar implica priorizar ciertas dimensiones
y dejar otras de lado: por eso, el modelo es perfectible y abierto a
ajustes según lo que surja en la experiencia concreta.

Algunas variables serán fácilmente identificables, otras pueden requerir
consensos grupales (por ejemplo, área temática o tipo de estudio) o
interpretación cuidadosa. El docente acompaña el proceso, pero los
equipos son responsables de justificar sus decisiones de extracción ante
dudas o ambigüedades.

\begin{quote}
\textbf{Nota:}\\
Si algún grupo identifica la necesidad de agregar o modificar variables
para mejorar la calidad del análisis, puede proponerlo y justificarlo en
conjunto. Así, el instrumento de extracción se convierte en un recurso
vivo, ajustado a la realidad del campo.
\end{quote}

\begin{center}\rule{0.5\linewidth}{0.5pt}\end{center}

\section{Síntesis y pasos
siguientes}\label{suxedntesis-y-pasos-siguientes}

El modelo de extracción y los criterios aquí definidos serán la base
sobre la que se organizará todo el trabajo del taller. El desafío
consiste no solo en aplicar la herramienta, sino en reflexionar
críticamente sobre sus alcances y limitaciones, proponiendo mejoras y
adaptaciones cuando sea necesario.

En el siguiente capítulo, se desarrollará una guía práctica paso a paso
para orientar a los equipos estudiantiles en cada etapa del proceso.

\begin{center}\rule{0.5\linewidth}{0.5pt}\end{center}

\bookmarksetup{startatroot}

\chapter{Guía práctica para
estudiantes}\label{guuxeda-pruxe1ctica-para-estudiantes}

Esta guía acompaña a cada equipo de estudiantes y al docente en todas
las etapas del taller. El trabajo se realizará de manera conjunta y
colaborativa: los docentes y los estudiantes participan a la par en la
toma de decisiones, la búsqueda, el análisis y la reflexión. Además, el
uso del \textbf{wiki} es clave desde el inicio: servirá para registrar
dudas, acuerdos, recomendaciones y aprendizajes, ayudando a construir un
legado útil para las futuras cohortes.

\begin{center}\rule{0.5\linewidth}{0.5pt}\end{center}

\section{¿Qué vamos a hacer?}\label{quuxe9-vamos-a-hacer}

El objetivo es conocer qué se ha investigado en fisioterapia y
kinesiología en América Latina y Brasil en los últimos años. Para
lograrlo, trabajaremos en equipos, acompañados por el docente, y
recorreremos juntos las etapas principales de una revisión sistemática
de alcance:

\begin{enumerate}
\def\labelenumi{\arabic{enumi}.}
\tightlist
\item
  Realizar una búsqueda de aproximación para encontrar palabras clave y
  términos relevantes.
\item
  Buscar artículos científicos relevantes en bases de datos confiables
  usando la estrategia final.
\item
  Seleccionar (filtrar) los estudios que cumplen los criterios
  definidos.
\item
  Leer críticamente los artículos seleccionados.
\item
  Extraer y registrar los datos más importantes de cada trabajo.
\item
  Analizar en conjunto los resultados y presentar los hallazgos.
\item
  Compartir dudas, soluciones y recomendaciones en el wiki colaborativo.
\end{enumerate}

\begin{center}\rule{0.5\linewidth}{0.5pt}\end{center}

\section{Pasos del proceso}\label{pasos-del-proceso}

\subsection{Paso 1: Búsqueda de aproximación y exploración de palabras
clave}\label{paso-1-buxfasqueda-de-aproximaciuxf3n-y-exploraciuxf3n-de-palabras-clave}

\begin{itemize}
\tightlist
\item
  Antes de realizar la búsqueda principal, realicen una \textbf{búsqueda
  de aproximación} en las bases de datos recomendadas (PubMed, SciELO,
  LILACS, etc.).
\item
  El objetivo es explorar qué términos aparecen en los títulos,
  resúmenes y palabras clave de los artículos recientes sobre el tema.
\item
  Tomen nota de los términos más frecuentes, sinónimos y palabras clave
  (en español, inglés y portugués si es posible).
\item
  Discutan en grupo cuáles parecen más útiles o relevantes para el
  objetivo del taller.
\item
  Registren en el wiki los términos elegidos y la lógica de la búsqueda.
  Así, otros equipos (y futuras cohortes) podrán mejorar sus búsquedas
  con la experiencia acumulada.
\item
  Una vez identificadas las mejores palabras clave, armen la estrategia
  de búsqueda final, combinando términos y usando los filtros adecuados
  (año, país, idioma).
\end{itemize}

\textbf{Ejemplo:}\\
Si buscan sobre ``rehabilitación en adultos mayores'', primero prueben
con términos como ``rehabilitación geriátrica'', ``elderly
rehabilitation'', ``adultos mayores'', ``aged'', y revisen qué términos
usan los artículos más relevantes en su resumen y sus palabras clave.

\textbf{Diferencia clave:}\\
La \textbf{búsqueda de aproximación} es flexible y exploratoria, ayuda a
definir cómo buscar mejor.\\
La \textbf{búsqueda sistemática} es la estrategia final, estructurada y
replicable, que aplican luego para encontrar todos los artículos
relevantes siguiendo los criterios acordados.

\subsection{Paso 2: Búsqueda sistemática y
registro}\label{paso-2-buxfasqueda-sistemuxe1tica-y-registro}

\begin{itemize}
\tightlist
\item
  Utilicen las palabras clave y la estrategia final definida en la
  búsqueda de aproximación.
\item
  Busquen artículos en las bases recomendadas.
\item
  Realicen búsquedas en equipos y comparen los resultados.
\item
  Registren cualquier dificultad o hallazgo interesante en el wiki.
\end{itemize}

\subsection{Paso 3: Selección y
filtrado}\label{paso-3-selecciuxf3n-y-filtrado}

\begin{itemize}
\tightlist
\item
  Lean títulos y resúmenes para decidir si el artículo cumple con los
  criterios de inclusión/exclusión.
\item
  Si hay dudas, discútanlas en grupo antes de decidir.
\item
  Registren los motivos de exclusión (si es posible) y cualquier
  dificultad en el wiki.
\end{itemize}

\subsection{Paso 4: Lectura crítica}\label{paso-4-lectura-cruxedtica}

\begin{itemize}
\tightlist
\item
  Lean el artículo completo, prestando atención a la metodología y al
  objetivo principal.
\item
  Identifiquen la población estudiada, el área temática y el tipo de
  estudio.
\item
  Anoten dudas o ambigüedades para discutirlas en equipo y documentarlas
  en el wiki.
\end{itemize}

\subsection{Paso 5: Extracción y registro de
datos}\label{paso-5-extracciuxf3n-y-registro-de-datos}

\begin{itemize}
\tightlist
\item
  Utilicen la tabla de extracción consensuada (ver Capítulo 2).
\item
  Completen cada campo con la información más precisa posible.
\item
  Si falta información, anótenlo como ``no reportado''.
\item
  Registren dificultades o acuerdos metodológicos en el wiki.
\end{itemize}

\subsection{Paso 6: Análisis y
presentación}\label{paso-6-anuxe1lisis-y-presentaciuxf3n}

\begin{itemize}
\tightlist
\item
  En grupo, revisen los datos extraídos y busquen patrones, tendencias y
  vacíos.
\item
  Elaboren gráficos o tablas simples para visualizar los resultados.
\item
  Preparen una síntesis para compartir en la clase final y en el wiki.
\end{itemize}

\begin{center}\rule{0.5\linewidth}{0.5pt}\end{center}

\section{Consejos prácticos y errores
frecuentes}\label{consejos-pruxe1cticos-y-errores-frecuentes}

\begin{itemize}
\tightlist
\item
  \textbf{No copies y pegues sin entender.} Resume con tus propias
  palabras siempre que puedas.
\item
  \textbf{No dejes para último momento la lectura de los artículos.}
  Distribuye el trabajo en tu equipo.
\item
  \textbf{Revisa que todos los campos de la tabla estén completos.} Las
  omisiones dificultan el análisis grupal.
\item
  \textbf{Consulta ante cualquier duda.} Es preferible preguntar a
  avanzar con errores.
\item
  \textbf{Registra desacuerdos o dificultades en el wiki.} Esto ayuda a
  las próximas cohortes y al equipo docente.
\end{itemize}

\begin{center}\rule{0.5\linewidth}{0.5pt}\end{center}

\section{Trabajo en equipo, comunicación y uso del
wiki}\label{trabajo-en-equipo-comunicaciuxf3n-y-uso-del-wiki}

\begin{itemize}
\tightlist
\item
  Define objetivos claros dentro del grupo (buscar, leer, completar la
  tabla, realizar registros en el wiki, etc.).
\item
  Coordinen encuentros regulares (presenciales o virtuales) para revisar
  el avance y compartir hallazgos.
\item
  Resuelvan los desacuerdos dialogando y, si es necesario, con apoyo
  docente.
\item
  Utilicen el wiki como espacio de registro vivo: escriban dudas,
  acuerdos, consejos y aprendizajes para dejar un legado a los futuros
  estudiantes.
\end{itemize}

\begin{center}\rule{0.5\linewidth}{0.5pt}\end{center}

\section{¿Qué hacer si surgen
problemas?}\label{quuxe9-hacer-si-surgen-problemas}

\begin{itemize}
\tightlist
\item
  Si no encuentran suficientes artículos, revisen los términos de
  búsqueda o consulten otras bases.
\item
  Si hay dificultades para decidir la inclusión o exclusión de un
  artículo, busquen el consenso grupal y, si persiste la duda, consulten
  al docente.
\item
  Si un miembro del grupo no participa, comuníquenlo a tiempo para
  evitar sobrecargas.
\item
  Cualquier situación particular puede y debe quedar registrada en el
  wiki.
\end{itemize}

\begin{center}\rule{0.5\linewidth}{0.5pt}\end{center}

\section{Glosario mínimo}\label{glosario-muxednimo}

\begin{itemize}
\tightlist
\item
  \textbf{Búsqueda de aproximación (exploratoria):}\\
  Primer rastreo flexible de artículos para identificar palabras clave,
  términos y estrategias de búsqueda. Sirve para diseñar la búsqueda
  sistemática.
\item
  \textbf{Búsqueda sistemática:}\\
  Estrategia final, precisa y replicable, con criterios claros y filtros
  definidos, para encontrar todos los artículos relevantes sobre el
  tema.
\item
  \textbf{Revisión sistemática de alcance:}\\
  Mapea qué y cómo se ha investigado en un campo temático, sin evaluar
  necesariamente la calidad de cada estudio.
\item
  \textbf{Criterios de inclusión/exclusión:}\\
  Condiciones para decidir qué artículos analizar y cuáles descartar.
\item
  \textbf{Extracción de datos:}\\
  Proceso de registrar información clave de cada estudio en una planilla
  común.
\item
  \textbf{Wiki:}\\
  Espacio colaborativo para compartir dudas, aprendizajes, acuerdos y
  recomendaciones.
\end{itemize}

\begin{center}\rule{0.5\linewidth}{0.5pt}\end{center}

\section{¿Cómo se evaluará tu
participación?}\label{cuxf3mo-se-evaluaruxe1-tu-participaciuxf3n}

La evaluación será integral y tendrá en cuenta tanto tu participación
durante el proceso como la calidad de los productos colaborativos. Estos
son los principales criterios:

\begin{itemize}
\tightlist
\item
  \textbf{Participación activa y sostenida:} Se espera que participes en
  las discusiones, en la toma de decisiones y en las tareas del grupo,
  tanto en clases como fuera de ellas.
\item
  \textbf{Aportes al trabajo de revisión:} Se valorarán tanto la
  cantidad como la calidad de tus contribuciones a la búsqueda,
  selección, análisis y registro de datos.
\item
  \textbf{Aportes al wiki colaborativo:} El registro de dudas,
  soluciones, recomendaciones y reflexiones en el wiki es fundamental.
  Tu experiencia puede ayudar a futuras cohortes.
\item
  \textbf{Presentismo:} La asistencia a las clases y actividades es
  importante para el funcionamiento del equipo.
\item
  \textbf{Informe final:} La calidad del informe de avance y la
  presentación de los resultados serán evaluadas por su claridad, orden
  y profundidad.
\end{itemize}

\textbf{Recuerda:}\\
El proceso es tan importante como el resultado final. El aprendizaje
colectivo, la colaboración y la honestidad intelectual se valoran tanto
como el producto escrito. No dudes en consultar, proponer mejoras o
registrar tus experiencias en el wiki.

\begin{center}\rule{0.5\linewidth}{0.5pt}\end{center}

\bookmarksetup{startatroot}

\chapter{Wiki del proceso colaborativo y legado
estudiantil}\label{wiki-del-proceso-colaborativo-y-legado-estudiantil}

El wiki es una herramienta central en este taller. No es solo un espacio
para ``dudas y preguntas frecuentes'', sino un \textbf{manual vivo}
donde cada cohorte de estudiantes deja registro de su experiencia,
errores, soluciones y recomendaciones para quienes vengan después.
Construir el wiki es, al mismo tiempo, aprender haciendo y dejar una
huella para la comunidad.

\begin{center}\rule{0.5\linewidth}{0.5pt}\end{center}

\section{¿Para qué sirve el wiki?}\label{para-quuxe9-sirve-el-wiki}

\begin{itemize}
\tightlist
\item
  \textbf{Compartir dudas y problemas:} Anotar cualquier obstáculo,
  confusión o dificultad encontrada durante el proceso.
\item
  \textbf{Registrar soluciones y acuerdos:} Escribir cómo resolvieron
  esas dificultades, qué consensos alcanzó el grupo o qué estrategias
  funcionaron.
\item
  \textbf{Recomendar buenas prácticas:} Dejar consejos y atajos útiles
  para buscar, seleccionar, leer o registrar artículos.
\item
  \textbf{Crear un legado:} Cada entrada suma a una base de conocimiento
  que será útil a los próximos equipos y a los propios docentes.
\end{itemize}

\begin{center}\rule{0.5\linewidth}{0.5pt}\end{center}

\section{¿Qué tipo de entradas puedo (y debo)
sumar?}\label{quuxe9-tipo-de-entradas-puedo-y-debo-sumar}

Algunas ideas para el tipo de información a dejar en el wiki:

\begin{itemize}
\tightlist
\item
  Errores frecuentes y cómo los resolvieron (por ejemplo: ``Nos
  confundimos al identificar la población y esto hicimos para
  solucionarlo\ldots{}'')
\item
  Dudas metodológicas y las respuestas encontradas (consultando
  docentes, leyendo o debatiendo)
\item
  Consejos para buscar artículos o elegir palabras clave
\item
  Recomendaciones sobre organización grupal
\item
  Sugerencias para el uso de la planilla de extracción o el manejo de
  referencias
\item
  Frases o reflexiones breves para futuras cohortes (``No subestimen la
  importancia de definir bien los criterios de inclusión\ldots{}'')
\end{itemize}

\begin{center}\rule{0.5\linewidth}{0.5pt}\end{center}

\section{¿Cómo se usa el wiki en Moodle? (Guía paso a
paso)}\label{cuxf3mo-se-usa-el-wiki-en-moodle-guuxeda-paso-a-paso}

\begin{enumerate}
\def\labelenumi{\arabic{enumi}.}
\tightlist
\item
  \textbf{Accede al aula virtual del taller y busca la sección Wiki}
  (puede llamarse ``Wiki colaborativa'', ``Registro de proceso'', etc.).
\item
  \textbf{Haz clic en ``Añadir entrada'' o ``Editar''} (según esté
  configurado).
\item
  \textbf{Escribe un título claro y breve} (por ejemplo: ``Cómo elegimos
  palabras clave'', ``Error frecuente: exclusión de artículos'').
\item
  \textbf{Redacta tu entrada de forma concisa y concreta}. Si es una
  duda, sé específico. Si es una recomendación, usa ejemplos o justifica
  el consejo.
\item
  \textbf{(Opcional) Adjunta una imagen o captura de pantalla} si es
  útil para ilustrar el problema o la solución.
\item
  \textbf{Guarda la entrada} y verifica que quede visible para el resto
  del grupo.
\item
  \textbf{Lee y comenta las entradas de otros} si tienes algo que sumar,
  aclarar o mejorar.
\item
  \textbf{No borres aportes de otros}: si una entrada ya fue respondida
  o actualizada, agrega un comentario debajo o crea una nueva versión.
\end{enumerate}

\begin{center}\rule{0.5\linewidth}{0.5pt}\end{center}

\subsection{4.3.1. ¿Cómo crear nuevas páginas, secciones y un índice en
el wiki de
Moodle?}\label{cuxf3mo-crear-nuevas-puxe1ginas-secciones-y-un-uxedndice-en-el-wiki-de-moodle}

\begin{itemize}
\tightlist
\item
  Para mantener la información organizada y accesible, puedes crear
  nuevas páginas dentro del wiki para cada tema, duda frecuente, error o
  recomendación.
\item
  \textbf{¿Cómo hacerlo?}

  \begin{enumerate}
  \def\labelenumi{\arabic{enumi}.}
  \tightlist
  \item
    Cuando estés editando o creando una entrada, escribe el título de la
    página que quieras crear entre dobles corchetes, por ejemplo:
    \texttt{{[}{[}Errores\ frecuentes{]}{]}} o
    \texttt{{[}{[}Palabras\ clave\ útiles{]}{]}}.\\
  \item
    Al guardar tu entrada, Moodle convertirá ese texto en un enlace a
    una nueva página, que podrás completar después.
  \item
    Haz clic en el enlace recién creado para editar esa nueva página y
    sumar contenido relacionado.
  \item
    Puedes anidar páginas, por ejemplo:
    \texttt{{[}{[}Errores\ frecuentes/Extracción\ de\ datos{]}{]}} para
    crear una subpágina.
  \end{enumerate}
\item
  \textbf{Sugerencia:}\\
  Crea una página principal (por ejemplo: ``Inicio del Wiki'' o ``Manual
  del taller'') con un índice de secciones y enlaces a cada página
  creada por el grupo. Así, el wiki será más fácil de navegar para
  todos.
\end{itemize}

\subsubsection{Ejemplo de índice wiki (Manual del
taller):}\label{ejemplo-de-uxedndice-wiki-manual-del-taller}

Puedes copiar y pegar esto en la \textbf{página principal} del wiki, y
los títulos entre corchetes crearán automáticamente los enlaces:

= Manual del taller: índice = {[}{[}Dudas metodológicas{]}{]}
{[}{[}Recomendaciones para buscar artículos{]}{]} {[}{[}Buenas prácticas
para el registro de datos{]}{]} {[}{[}Errores frecuentes{]}{]} {[}{[}FAQ
del taller{]}{]}

\begin{quote}
\textbf{Tip:} Si tienes dudas sobre cómo usar las funciones del wiki en
Moodle, pide una demostración al docente o revisa los
\href{https://docs.moodle.org/all/es/Uso_de_Wiki}{tutoriales cortos} de
la plataforma.
\end{quote}

\begin{center}\rule{0.5\linewidth}{0.5pt}\end{center}

\subsection{¿Cómo se ve una entrada de
wiki?}\label{cuxf3mo-se-ve-una-entrada-de-wiki}

\textbf{Esqueleto recomendado:}

\begin{itemize}
\tightlist
\item
  \textbf{Título:} Describe el tema en pocas palabras.
\item
  \textbf{Problema o duda:}\\
  Explica brevemente el problema encontrado, la situación concreta o la
  pregunta que surgió.
\item
  \textbf{Solución, respuesta o recomendación:}\\
  Describe cómo lo resolviste (o cómo te lo explicaron). Si es una
  recomendación, justifica brevemente por qué puede ayudar a otros.
\item
  \textbf{(Opcional) Ejemplo real o breve captura:}\\
  Si corresponde, agrega una imagen o una cita breve.
\item
  \textbf{Reflexión final (opcional):}\\
  ¿Qué aprendiste de la experiencia? ¿Qué sugerirías a los futuros
  equipos sobre este tema?
\end{itemize}

\textbf{Ejemplo completo de entrada útil:}

\begin{itemize}
\tightlist
\item
  \textbf{Título:} Dificultad para encontrar palabras clave en LILACS
\item
  \textbf{Problema:}\\
  No encontrábamos artículos relevantes usando solo ``fisioterapia'' y
  ``rehabilitación''. Los resultados eran muy generales.
\item
  \textbf{Solución:}\\
  Exploramos los artículos que sí eran relevantes y encontramos que
  muchos usaban el término ``movimiento funcional'' y ``rehabilitación
  motora''. Agregamos estos términos y filtramos por país.
\item
  \textbf{Reflexión:}\\
  Sugerimos siempre hacer una búsqueda de aproximación antes de definir
  la estrategia final y anotar los términos más productivos en el wiki.
\end{itemize}

\begin{center}\rule{0.5\linewidth}{0.5pt}\end{center}

\section{Dudas más frecuentes (FAQ)}\label{dudas-muxe1s-frecuentes-faq}

\begin{itemize}
\item
  \textbf{¿Qué hago si dos artículos parecen casi iguales?}\\
  Verifica si son el mismo estudio publicado en dos revistas o si hay
  diferencias reales (diseño, población, etc.). Consulta al docente ante
  la duda y registra el caso en el wiki.
\item
  \textbf{¿Cómo decidir si un artículo cumple los criterios de inclusión
  si no hay suficiente información en el resumen?}\\
  Intenta acceder al texto completo. Si sigue sin quedar claro,
  discútelo en grupo y consúltalo con el docente.
\item
  \textbf{¿Es obligatorio que todos los miembros del grupo participen en
  el wiki?}\\
  Sí, la participación es colectiva y todos los aportes son valiosos.
\item
  \textbf{¿Qué pasa si me equivoco al registrar algo en el wiki?}\\
  Puedes corregirlo en una nueva entrada o comentario, pero nunca borres
  lo que ya está (así queda registro de todo el proceso).
\item
  \textbf{¿Qué tipo de recomendaciones son útiles para futuras
  cohortes?}\\
  Todo lo que te hubiera gustado saber al empezar: atajos, errores
  evitables, dudas resueltas, estrategias de organización.
\end{itemize}

\begin{center}\rule{0.5\linewidth}{0.5pt}\end{center}

\section{Por qué dejar un legado:}\label{por-quuxe9-dejar-un-legado}

\begin{itemize}
\tightlist
\item
  Tu experiencia puede ayudar a futuros estudiantes a evitar los mismos
  errores o resolver dudas más rápido.
\item
  El wiki también es útil para docentes: les muestra las verdaderas
  dificultades, aciertos y necesidades de cada grupo.
\item
  Un taller con un wiki robusto es más ágil y enriquecedor para todos.
\end{itemize}

\textbf{Recuerda:}\\
Cada entrada, por mínima que sea, puede hacer la diferencia para otros.
Lo que hoy es tu duda, mañana puede ser la respuesta clave de un equipo
nuevo.

\begin{center}\rule{0.5\linewidth}{0.5pt}\end{center}

\bookmarksetup{startatroot}

\chapter{Cierre: hacia una comunidad de aprendizaje e
investigación}\label{cierre-hacia-una-comunidad-de-aprendizaje-e-investigaciuxf3n}

Este documento es solo un punto de partida. Tanto el taller como los
materiales aquí presentados fueron pensados para evolucionar junto con
las necesidades, intereses y desafíos de quienes lo utilicen. Su valor
radica en la apropiación colectiva, la actualización constante y la
disposición a repensar los métodos, las preguntas y los enfoques.

La construcción del conocimiento en salud exige apertura, rigor y
colaboración. Esperamos que la experiencia de revisar, analizar y
producir colectivamente marque una diferencia real en la formación, y
también inspire a quienes sigan este camino a buscar nuevas preguntas y
soluciones.

\section{Posibles líneas futuras}\label{posibles-luxedneas-futuras}

\begin{itemize}
\tightlist
\item
  Ampliar a áreas temáticas emergentes.
\item
  Sumar herramientas de análisis cualitativo y visualización avanzada de
  datos.
\item
  Integrar este modelo de taller con actividades de extensión o
  proyectos interdisciplinarios.
\item
  Desarrollar espacios de discusión sobre ética y calidad en la
  investigación.
\item
  Publicar resultados en repositorios abiertos o como preprints.
\end{itemize}

El aprendizaje nunca termina: cada cohorte puede dejar nuevas preguntas,
herramientas y recomendaciones, mejorando la experiencia para quienes
vengan después. El documento y el wiki son herramientas vivas: dependen
del compromiso y la creatividad de cada estudiante y docente.

\begin{center}\rule{0.5\linewidth}{0.5pt}\end{center}


\backmatter


\end{document}
